\documentclass[11pt,letterpaper]{article}
%\usepackage[utf8]{inputenc}
%\usepackage{fontspec}
%\usepackage{listingsutf8}
\usepackage[margin=1in]{geometry}
\usepackage{graphicx}
\usepackage{pgffor}
\usepackage{amsmath}
\usepackage{amssymb}
\usepackage{fancyhdr}
\usepackage{hyperref}
\usepackage{subcaption}
\usepackage{bbm}
\usepackage{lscape}


% ------------------------------------------------------

\DeclareMathOperator*{\argmin}{arg\,min}
\DeclareMathOperator*{\argmax}{arg\,max}
\DeclareMathOperator{\Var}{Var}
\DeclareMathOperator{\Cov}{Cov}
\def\inprob{\,{\buildrel p \over \longrightarrow}\,} 
\def\indist{\,{\buildrel d \over \longrightarrow}\,} 

\DeclareMathOperator\F{\mathcal{F}}

% ------------------------------------------------------
 
\input{./julia-listing.tex}

\graphicspath{{../plots/pdf/}{../plots/}}
\setlength\parindent{0.0in}

% ------------------------------------------------------

\title{\textbf{Homework 7} \\ Labor Economics}
\author{Mark Agerton}
\date{Due Mon, Feb 23}

% ------------------------------------------------------

\begin{document}

\maketitle

\section{Setup}
Wages are
\begin{align*}
y_0 &= \delta_0 + \beta_0 x + \theta + \epsilon_0 \\
y_1 &= \delta_1 + \beta_1 x + \alpha_1\theta + \epsilon_1
\end{align*}
Also define the utility shifter function $C$ and an index function $I$
\begin{align*}
C &= \gamma_0 + \gamma_2 z + \gamma_3 x + \alpha_C \theta  \\
I &= E[y_1 - y_0 - C|\F]  \\
  &= \underbrace{(\delta_1 - \delta_0 - \gamma_0)}_{\widetilde \delta} + \underbrace{(\beta_0 - \beta_1 - \gamma_3)}_{\widetilde \beta} x_i - \gamma_2 z + \underbrace{(\alpha_1 - 1- \alpha_C)}_{\widetilde \alpha} \theta - \epsilon_{c}
\end{align*}
The distribution of shocks is
\[
\left. \begin{pmatrix} \epsilon_{i,0} \\ \epsilon_{i,1} \\ \epsilon_{i,C} \end{pmatrix} \right|_{x_i,z_i,\theta_i}
\sim N
\left[ \begin{pmatrix} 0 \\ 0 \\ 0 \end{pmatrix}, \begin{pmatrix} \sigma_0^2& 0 & 0\\0& \sigma_1^2& 0 \\0 & 0& \sigma_C^2\end{pmatrix}	\right] 
\]
The information set for the agent is $\F$. The preference shock $\epsilon_C \in \F$, but $\{\epsilon_0,\epsilon_1\} \notin \F$. The decision rule is
\[
s = 1 \Longleftrightarrow E[I \geq 0 | \F]
\]

\section{Q1}
There is no unobserved heterogeneity in this model since we know $\theta$. Thus,
\[
E\left[ \begin{matrix} y_{0} \\ y_{1} \end{matrix} \middle | x_i,\theta_i, s=k \right]
=
\begin{bmatrix} 
	\delta_0 + x\beta_0 + \theta         \\  
	\delta_1 + x\beta_1 + \alpha_1\theta 
\end{bmatrix} + 
\underbrace{E\left[ \begin{matrix} \epsilon_{0} \\ \epsilon_{1} \end{matrix} \middle | x_i,\theta_i, \epsilon_c:big/small \right]}_{0}
\]
This is straight-up OLS, which means we recover $\{\delta, \beta, \alpha_1, \sigma^2_0, \sigma^2_1\}$. 
\begin{align*}
\Pr[S=1|\F] 
&= \Pr\left[\epsilon_c \leq \widetilde \delta + \widetilde \beta x - \gamma_2 z + \widetilde \alpha \theta \middle | \F\right] \\
&= \Phi\left[\frac{\overbrace{\left[(\delta_1 - \delta_0) + (\beta_1-\beta_0)x + (\alpha_1 - 1)\theta\right]}^{\text{known number}} -\gamma_0 - \gamma_2 z - \gamma_3 x - \alpha_c \theta }{\sigma_c} \middle | \F\right]
\end{align*}
Now we can get $\left\{ \gamma_0, \gamma_2, \gamma_3, \alpha_c, \sigma_c^2 \right\}$


\section{Q2}

Now we don't know $\theta$ but agents do. However, we do have two measurement equations $m \in \{A,B\}$:
\begin{align*}
M_{iA} &= x_i^M \beta_A^M + \theta_i + \epsilon_{iA}^M \\
M_{iB} &= x_i^M \beta_B^M + \alpha_B\theta_i + \epsilon_{iB}^M
\end{align*}
where $\epsilon_m^M \sim N(0,\sigma^{M2}_m)$ are i.i.d. We can write
\begin{align*}
E[y_1 | x,z,s=1] 
&= \delta_1 + \beta_1 x + E[\epsilon_1 + \alpha_1\theta | x,z, I\geq 0] \\
&= \delta_1 + \beta_1 x + \alpha_1 E[\theta | x,z, I\geq 0] \\
&= \delta_1 + \beta_1 x + \alpha_1\sigma^* E\left[\frac{\theta}{\sigma^*} \middle|0 \leq \widetilde \delta + \widetilde \beta x - \gamma_2 z + \underbrace{(\alpha_1 - \alpha_0 - \alpha_c)\theta - \epsilon_c}_\eta  \right] \\
&= \delta_1 + \beta_1 x + \alpha_1\sigma^* E\left[\frac{\theta}{\sigma^*} \middle| \eta \geq - \left( \widetilde \delta + \widetilde \beta x - \gamma_2 z  \right) \right]
\end{align*}
Define $\eta \equiv (\alpha_1 - \alpha_0 - \alpha_c)\theta - \epsilon_c$. Then 
\[
\begin{pmatrix} \eta \\ \theta \end{pmatrix} 
\sim N
\left[ 
	\begin{pmatrix} 0 \\ 0  \end{pmatrix}, 
	\begin{pmatrix} 
		\sigma^{*2}     & (\alpha_1 - \alpha_0 - \alpha_c)\sigma^2_\theta \\
		(\alpha_1 - \alpha_0 - \alpha_c)\sigma^2_\theta & \sigma^2_\theta \\
	\end{pmatrix}
\right] 
\]
where $\sigma^{*2} = (\alpha_1 - \alpha_0 - \alpha_c)^2\sigma^2_\theta + \sigma^2_c$. We can project $\theta$ onto $\eta$, which means
\[
\theta = \frac{\Cov(\eta,\theta)}{\Var \eta } \eta + \nu = \frac{(\alpha_1 - \alpha_0 - \alpha_c)\sigma^2_\theta}{\sigma^{*2}}\eta + \nu
\]
where
\[
\nu \sim N\left( 0, \sigma^2_\theta \left( 1- \rho_{\eta\theta}^2\right)\right)
\qquad\text{and} \qquad
\rho_{\eta\theta} = \frac{(\alpha_1 - \alpha_0 - \alpha_c)\sigma^2_\theta}{\sigma^{*}\sigma_\theta}
\]
Letting $t \equiv -(\widetilde \delta + \widetilde \beta x - \gamma_2 z)/\sigma^* $, we can now write
\begin{align*}
E[y_0 | x,z,s=1] 
&= \delta_1 + \beta_0 x + \alpha_0 \frac{(\alpha_1 - \alpha_0 - \alpha_c)\sigma^2_\theta}{\sigma^*} \overbrace{\frac{-\phi(t)}{\Phi(t)}}^{\lambda_0} \\
E[y_1 | x,z,s=1] 
&= \delta_1 + \beta_1 x + \alpha_1 \frac{(\alpha_1 - \alpha_0 - \alpha_c)\sigma^2_\theta}{\sigma^*} \underbrace{\frac{\phi(t)}{1-\Phi(t)}}_{\lambda_1} 
\end{align*}


\section{Histograms}
\begin{landscape}
\begin{figure}[h]
	\centering
	\begin{subfigure}[b]{0.3\textwidth}\centering \includegraphics[width=1\textwidth]{x}  \end{subfigure}
	\begin{subfigure}[b]{0.3\textwidth}\centering \includegraphics[width=1\textwidth]{z}  \end{subfigure}
    \begin{subfigure}[b]{0.3\textwidth}\centering \includegraphics[width=1\textwidth]{ma} \end{subfigure}

    \begin{subfigure}[b]{0.3\textwidth}\centering \includegraphics[width=1\textwidth]{mb} \end{subfigure}
    \begin{subfigure}[b]{0.3\textwidth}\centering \includegraphics[width=1\textwidth]{xm} \end{subfigure}
    \begin{subfigure}[b]{0.3\textwidth}\centering \includegraphics[width=1\textwidth]{y} \end{subfigure}    
    \caption{Plots}		
\end{figure}	
\end{landscape}


\end{document}


